\documentclass{beamer}
\usepackage{pifont}
\begin{document}

\title{Leveraging version control to accelerate research}
\subtitle{or Lessons learned in academic software engineering\\ \small{or Don't make the mistakes I did!}}
\author{Criston Hyett}

\graphicspath{{./figs}}
% ---------------------------------------------------------------------------%
\begin{frame}
  \maketitle

\end{frame}
% ---------------------------------------------------------------------------%

% ---------------------------------------------------------------------------%
\begin{frame}
  \tableofcontents
\end{frame}
% ---------------------------------------------------------------------------%


\section{Introduction}
% ---------------------------------------------------------------------------%
\begin{frame}{A note on terminology}
  Github $\subset$ git $\subset$ Version Tracking/Control $\subset$ Configuration Management
\end{frame}
% ---------------------------------------------------------------------------%

% ---------------------------------------------------------------------------%
\begin{frame}{Introduction to version control}
  \textbf{Version Control:} In software engineering, version control is a class of systems responsible for \textbf{managing changes to} computer programs, documents, large web sites, or other \textbf{collections of information.}
\only<2>{
  \begin{equation*}
    \downarrow
  \end{equation*}

  \textbf{Version control manages changes to collections of information.}}
\end{frame}
% ---------------------------------------------------------------------------%

% ---------------------------------------------------------------------------%
\begin{frame}{Introduction to version control}
  \begin{itemize}
  \item Broadly, version control allows you to:
    \begin{itemize}
    \item track, manage, and understand changes
    \item collaborate effectively
    \item store backups
    \item create and build upon ``known states''
    \end{itemize}
    \only<1>{
    \item Version control is often seen as a safety net, which may be why it is often ignored.}
    \only<2>{
    \item But perhaps we should view it more as a trampoline!
    }
  \end{itemize}

  \only<1>{
  \begin{figure}
    \centering
    \includegraphics[width=0.75\textwidth]{safety_net.jpg}
    \caption{By Stead's Review, Public Domain, \url{https://commons.wikimedia.org/w/index.php?curid=3813831}}
  \end{figure}
}

\only<2>{
  \begin{figure}
    \centering
    \includegraphics[width=0.75\textwidth]{trampoline.png}
  \end{figure}
  }
\end{frame}
% ---------------------------------------------------------------------------%
\begin{frame}{Introduction to version control}
  \only<1>{
  However, version control can also accelerate progress!...Well, indirectly at least.}

  I will argue that effective utilization can
  \begin{itemize}
    \only<1->{
    \item \textbf{Reduce cognitive overhead}
      
    Utilizing \textbf{pull requests} and \textbf{branches} lowers cognitive overhead to start/continue one or multiple projects.}

  \only<2->{
  \item \textbf{Deliver incremental success}
    
    Working from a \textit{known state} towards a \textit{known state} allows for better project definitions, particularly when a task/project is ``done''}

  \only<3->{
  \item \textbf{Reduce technical debt}
    
    Standardized, automated, decentralized testing bakes in reproducibility, creates known states, and alerts you of unintentional changes.}

  \only<4->{
  \item \textbf{Bake in tenants of Open Science from the start}

    Building a ``product'' that is incrementally constructed, continuously tested, serving a well-defined goal, reduces work at the end of a project and makes your work easier to understand and build on top of.}
    
  \end{itemize}
\end{frame}
% ---------------------------------------------------------------------------%

\section{Git \& Github}
% ---------------------------------------------------------------------------%
\begin{frame}{Introduction to git}
  \begin{itemize}
  \item {\color{blue} \href{https://git-scm.com/book/en/v2/Getting-Started-A-Short-History-of-Git}{git was introduced in 2005 to manage development of the linux kernel.}}

  \item Extremely lightweight, pervasive, effectively manages projects of any size.

  \item There are plenty of ways to use git - I will show one that is probably sufficient.
  \end{itemize}
\end{frame}
% ---------------------------------------------------------------------------%

% ---------------------------------------------------------------------------%
\begin{frame}{Introduction to github}
  Github is:
  \begin{itemize}
  \item Perfect in almost every way \ding{51}
  \item ...owned by Microsoft \ding{55}
  \end{itemize}
\end{frame}
% ---------------------------------------------------------------------------%

% ---------------------------------------------------------------------------%
\begin{frame}{Introduction to github}
  Github is a cloud-based service that
  \begin{itemize}
  \item \textbf{is free for students} (and many others)
  \item \textbf{synchronizes git repositories}

    Collaborate with others or yourself, and/or deploy to cloud computing and/or HPC
    
  \item \textbf{integrates with many toolchains} I alluded to earlier (continuous integration, documentation, project management, etc)

  \item Has a wonderful {\color{blue} \href{https://docs.github.com/en/get-started/start-your-journey}{getting started tutorial}}
  \end{itemize}
\end{frame}
% ---------------------------------------------------------------------------%


% ---------------------------------------------------------------------------%
\begin{frame}{The git(hub) workflow \& branches}
    \begin{figure}
    \centering
    \includegraphics[width=\textwidth]{github_flow.png}
  \end{figure}

  \begin{itemize}
  \item \textbf{Branch}: An isolated environment to make changes in, without effecting other branches
  \item \textbf{Commit}: A single batch of changes, usually quite small
  \item \textbf{Pull Request}: (PR) A formal request to merge two branches, usually consisting of peer review/tests/documentation of changes
  \end{itemize}
\end{frame}
% ---------------------------------------------------------------------------%

% ---------------------------------------------------------------------------%
\begin{frame}{Use cases}
  \begin{itemize}
    \only<1>{
    \item \textbf{Branches}: topic branch, development branch, bug-fix, integration test}

    \only<2>{
    \item \textbf{Commits}: commits should happen frequently. Finish that function? Commit. Get another test to pass? Commit. Prove that theorem? Commit.}

    \only<3>{
  \item \textbf{Pull Requests}: PRs should happen when a coherent block of functionality is complete. Fix the bug you were hunting? Open a PR. Finish the task? Open a PR.

    \vspace{0.5cm}

    Further, PRs are a great place to document the state of the work and handle review comments. Add figures, justify the approach taken, document any hang ups and review comments/follow ups.}
  \end{itemize}
\end{frame}
% ---------------------------------------------------------------------------%

% ---------------------------------------------------------------------------%
\begin{frame}{Quick example}
  
\end{frame}
% ---------------------------------------------------------------------------%

% ---------------------------------------------------------------------------%
\begin{frame}{Github Actions}
  Github actions are configurable, automated scripts that
  \begin{itemize}
  \item run at pre-determined times (e.g., on PRs/merges)
  \item automate process like continuous integration (unit/integration testing), build and deploy documentation, codecoverage analysis, etc
  \item can post results in PRs upon completion
  \end{itemize}
  \vspace{0.5cm}
  
  Example: You develop on your laptop and use github to sync to the HPC. Use github actions to ensure your code builds in a clean linux environment.
  
  \vspace{0.5cm}
  
  \only<2>{\textbf{Github actions are \textit{powerful}}}
  
\end{frame}
% ---------------------------------------------------------------------------%

% ---------------------------------------------------------------------------%
\begin{frame}{Github Actions}
  \begin{itemize}
  \item Actions are defined via a *.yml file (YAML)
  \item Via the YAML file, you specify
    \begin{itemize}
    \item triggers
    \item platform to build on
    \item build steps
    \item run steps
    \item post-run steps
    \end{itemize}
  \item Look at established repositories for examples!!
  \end{itemize}
\end{frame}
% ---------------------------------------------------------------------------%

% ---------------------------------------------------------------------------%
\begin{frame}{Github Actions Examples}
  
\end{frame}
% ---------------------------------------------------------------------------%

\section{Version control as a springboard}
% ---------------------------------------------------------------------------%
\begin{frame}{Version control as a springboard}
  \begin{itemize}
  \item \textbf{Reduce cognitive overhead}
    
  \item \textbf{Deliver incremental success}    

  \item \textbf{Reduce technical debt}
    
  \item \textbf{Bake in tenants of Open Science from the start}

  \end{itemize}
\end{frame}
% ---------------------------------------------------------------------------%

% ---------------------------------------------------------------------------%
\begin{frame}{Reduce cognitive overhead}
  \begin{itemize}
  \item Trying to solve your problem in one step is admittedly hard, if not impossible
  \item We're often told to break the problem down into subproblems
  \item This is usually done informally, if at all.
  \item Further, when working on a subproblem, its easy to get distracted by an urgent request, or a sticky sub-subproblem.
  \item \textbf{Use issues, branches and PRs to enforce doing one thing at a time!} Not only will this improve the quality of your solutions, but it gives you a fighting chance to understand and document the effects of your solution to the subproblem.
  \end{itemize}
\end{frame}
% ---------------------------------------------------------------------------%

% ---------------------------------------------------------------------------%
\begin{frame}{Deliver incremental success}
  \begin{itemize}
  \item What is a \textbf{known state}?
  \item Transitioning from a known state, to a known state:
    \begin{itemize}
    \item ensures performance is understood
    \item gives a posteriori clarity (\textit{Why did we make that decision?})
    \item builds trust, and allows the next, next step.
    \item enables test-driven development (or hypothesis-driven science)
    \end{itemize}
  \end{itemize}
\end{frame}
% ---------------------------------------------------------------------------%

% ---------------------------------------------------------------------------%
\begin{frame}{Reduce (and quantify) technical debt}
  \begin{itemize}
  \item Enforcing an automatic process against yourself is upfront work to stop technical debt accumulation!
  \item Post issues against your own code! What gets tracked gets managed!
  \item Test your code early and often - bonus points if its automated as part of your workflow
  \end{itemize}
\end{frame}
% ---------------------------------------------------------------------------%


% ---------------------------------------------------------------------------%
\begin{frame}{Bake in tenants of Open Science}
  These are all nice benefits for a worker, but as an advertiser, more rigorous process enforcement during research
  \begin{itemize}
  \item accelerates time to publication (less to clean up, fewer open questions, etc)
  \item makes sharing and collaborating \textit{much} easier
  \item builds trust in the product from reviewers, colleagues $\to$ citations!
  \end{itemize}
\end{frame}
% ---------------------------------------------------------------------------%


\section{Super Exciting Bonus Surprise}
% ---------------------------------------------------------------------------%
\begin{frame}{Templates!}
  \begin{itemize}
  \item Julia template project: \url{https://github.com/cmhyett/JuliaTemplate}

    Runs tests, builds documentation, calculates code coverage, looks for any nasty compatibility issues, enforces standardized formatting, includes (what I think are) helpful templates for issues and PRs.

  \item Github project template for project launch: \url{https://github.com/users/cmhyett/projects/3}

    Enforce hypothesis-driven science! Scope your work, get to publication faster and (maybe) with a more defined contribution.
  \end{itemize}
\end{frame}
% ---------------------------------------------------------------------------%

% ---------------------------------------------------------------------------%
\begin{frame}{Conlusions}
  \begin{itemize}
  \item In the middle of a project? That's fine - \textit{Do no harm}
  \item The only way to learn your best practices are to do them! Try a workflow! Modify your workflow till you balance load and payoff.
  \item If you find you like this, try to get your team to buy in. Trying to do the process for a team, by yourself is exhausting, and likely duplicates your meetings.
  \end{itemize}
\end{frame}
% ---------------------------------------------------------------------------%

\end{document}
